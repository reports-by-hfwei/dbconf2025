% rp-mil.tex

%%%%%%%%%%%%%%%%%%%%
\begin{frame}{}
	\begin{center}
		\blue{RePMILA:arXiv2025}
		\fig{width = 0.80\textwidth}{figs/rp-mil-arXiv2025}
		\begin{columns}
			\column{0.50\textwidth}
				\pause
				\fig{width = 0.90\textwidth}{figs/repmila-program}
			\column{0.50\textwidth}
				\pause
				\fig{width = 0.90\textwidth}{figs/repmila-program}
		\end{columns}
		\pause
		\vspace{0.30cm}
		\red{Not support for predicate reads, \textsc{Insert}, \textsc{Delete}}
	\end{center}
\end{frame}
%%%%%%%%%%%%%%%%%%%%

% %%%%%%%%%%%%%%%%%%%%
% \begin{frame}{arXiv2025 2501.18377v3 Using Read Promotion and Mixed Isolation Levels for Performant Yet Serializable Execution of Transaction Programs}
% 	\textbf{Problems: } 许多 DBMS 默认使用 SI 等较低的隔离级别,以换取更高的性能。但这将处理并发异常(如写偏斜)的责任推给了应用程序开发者。目前的挑战在于,如何在不牺牲性能的前提下,为事务程序提供可串行化的正确性保证,从而简化开发。

% 	\textbf{Contributions: }一种在多版本数据库中系统性地分配隔离级别给事务的框架,能够根据事务行为,自动分配最合适的隔离级别,在确保在所需正确性的前提下最大化性能。
% \end{frame}
% %%%%%%%%%%%%%%%%%%%%