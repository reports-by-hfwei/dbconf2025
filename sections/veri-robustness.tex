% robustness.tex

%%%%%%%%%%%%%%%%%%%%
\begin{frame}{}
	数据库鲁棒性常被分为动态鲁棒性和静态鲁棒性两个层次来讨论。

	\begin{itemize}
		\item 动态鲁棒性研究数据库的某一个弱隔离级别 $I_1$ 下的执行历史 $H$ 是否满足较强的隔离级别 $I_2$(一般是 SER)。如果满足,我们说 $H$ 在 $I_1$ 下是鲁棒的。
		\item 静态鲁棒性研究客户程序 $P$ 在弱隔离级别 $I_1$ 下的所有执行历史是否都满足较强的隔离级别 $I_2$(一般是 SER)。如果满足,我们说 $P$ 在 $I_1$ 下是鲁棒的。例如,如果 $P$ 在 SI 下鲁棒,则我们可以在 SI 下运行 $P$,这可以在保证执行历史满足 SER 的情况下提高系统性能和吞吐率。
	\end{itemize}
\end{frame}
%%%%%%%%%%%%%%%%%%%%

%%%%%%%%%%%%%%%%%%%%
\begin{frame}{}
	\begin{table}[]
\centering
\renewcommand{\arraystretch}{1.8}
\resizebox{\textwidth}{!}{%
\begin{tabular}{|c|c|c|c|c|c|}
	\hline
	$I_1 \setminus I_2$ & \pc & \si & \ser & \multicolumn{1}{c|}{15-20} & \multicolumn{1}{c|}{总数}
	\\ \hline
	\ru  & & & {[}Ketsman PODS'20{]} (非分布式)充要条件 & \multicolumn{1}{c|}{3} & \multicolumn{1}{c|}{15}
	\\ \hline
	\rc  & & & {[}Ketsman PODS'20{]} (非分布式)充要条件 {[}Vandevoort VLDB'21{]} (非分布式)事务模板的充要条件 & \multicolumn{1}{c|}{0} & \multicolumn{1}{c|}{10}
	\\ \hline
	\ra  & & & & \multicolumn{1}{c|}{3} & \multicolumn{1}{c|}{25}
	\\ \hline
	\tcc  & {[}Bouajjani ESOP'21{]} 充要条件,转化为 CC 对 SER 鲁棒性 & {[}Bouajjani ESOP'21{]} 充要条件,CC \textbackslash{}to PC \textbackslash{}land PC \textbackslash{}to SI & {[}Bernardi CONCUR'16{]} 充分条件 {[}Beillahi CONCUR'19{]} CC、CM、CCv 下充要条件                       &                            &
	\\ \hline
	\pc  & / & {[}Bouajjani ESOP'21{]} 充要条件,依赖程序搜索 & {[}Bernardi CONCUR'16{]} 充分条件 & &
	\\ \hline
	\psiso & / & {[}Cerone PODC'16{]} 充分条件 & {[}Bernardi CONCUR'16{]} 充分条件 & &
	\\ \hline
	\si  & / & /                                                                                                   & {[}Fekete TODS'05{]} 充分条件 {[}Cerone PODC'16 \& JACM'18{]} 充要条件 {[}Bernardi CONCUR'16{]} 充分条件 &                            &                         \\ \hline
\end{tabular}
}
\end{table}
\end{frame}
%%%%%%%%%%%%%%%%%%%%

%%%%%%%%%%%%%%%%%%%%
\begin{frame}{TODS2025 Allocating Isolation Levels to Transactions in a Multiversion Setting}
	\begin{columns}
		\begin{column}{0.5\textwidth}
			\textbf{Problems: }开发人员手动为每类的事务选择隔离级别具有挑战性。选择过于保守,会严重影响吞吐量;选择较低的隔离级别,则需要复杂的并发逻辑来避免异常。

			\textbf{Contributions: }一种在多版本数据库中系统性地分配隔离级别给事务的框架,能够根据事务行为,自动分配最合适的隔离级别,在确保在所需正确性的前提下最大化性能。
		\end{column}
		\begin{column}{0.5\textwidth}
			\includegraphics[width=0.98\linewidth]{figs/lowest-robust-allocations}
		\end{column}
	\end{columns}
\end{frame}
%%%%%%%%%%%%%%%%%%%%

%%%%%%%%%%%%%%%%%%%%
\begin{frame}{TODS2025 Allocating Isolation Levels to Transactions in a Multiversion Setting}
	\begin{columns}
		\begin{column}{0.5\textwidth}
			\textbf{Basic Ideas: }分析事务的访问模式(读集和写集),动态识别潜在的冲突依赖环:
			\begin{itemize}
				\item 对只读事务,安全分配 RC 以实现高吞吐量;
				\item 对可能产生依赖环的事务,自动提升隔离级别至 SI 或 SER;
				\item 结合静态分析与运行时监控,在事务执行前/期间动态决策隔离级别。
			\end{itemize}
		\end{column}
		\begin{column}{0.5\textwidth}
			\includegraphics[width=0.98\linewidth]{figs/confliction-detect}
		\end{column}
	\end{columns}
\end{frame}
%%%%%%%%%%%%%%%%%%%%

%%%%%%%%%%%%%%%%%%%%
\begin{frame}{TODS2025 Allocating Isolation Levels to Transactions in a Multiversion Setting}

	\textbf{Limitations: }
	\begin{itemize}
		\item 依赖调度分析,不支持分布式数据库;
		\item 执行前的分析引入额外的开销;
		\item 在高度冲突的工作负载下性能优势减弱。
	\end{itemize}

\end{frame}
%%%%%%%%%%%%%%%%%%%%

%%%%%%%%%%%%%%%%%%%%
\begin{frame}{arXiv2025 2501.18377v3 Using Read Promotion and Mixed Isolation Levels for Performant Yet Serializable Execution of Transaction Programs}
	\textbf{Problems: } 许多 DBMS 默认使用 SI 等较低的隔离级别,以换取更高的性能。但这将处理并发异常(如写偏斜)的责任推给了应用程序开发者。目前的挑战在于,如何在不牺牲性能的前提下,为事务程序提供可串行化的正确性保证,从而简化开发。


	\textbf{Contributions: }一种在多版本数据库中系统性地分配隔离级别给事务的框架,能够根据事务行为,自动分配最合适的隔离级别,在确保在所需正确性的前提下最大化性能。
\end{frame}
%%%%%%%%%%%%%%%%%%%%

%%%%%%%%%%%%%%%%%%%%
\begin{frame}{VLDB2025 TxnSails-Achieving Serializable Transaction Scheduling with Self-Adaptive Isolation Level Selection}
	\begin{columns}
		\begin{column}{0.5\textwidth}
			\textbf{Contributions: }
			\begin{itemize}
				\item 实现低隔离级别下的可串行化:在运行时细粒度地跟踪单个事务的读写依赖,并通过调度提交顺序(或必要时中止事务)来确保提交顺序与依赖顺序一致。相比静态修改 SQL(如加锁升级),这种方法减少了不必要的冲突和开销。
			\end{itemize}
		\end{column}
		\begin{column}{0.5\textwidth}
			\includegraphics[width=0.98\linewidth]{figs/TxnSails-txn-proc}
		\end{column}
	\end{columns}
\end{frame}
%%%%%%%%%%%%%%%%%%%%

%%%%%%%%%%%%%%%%%%%%
\begin{frame}{VLDB2025 TxnSails-Achieving Serializable Transaction Scheduling with Self-Adaptive Isolation Level Selection}
	\textbf{Contributions: }
	\begin{itemize}
		\item 面向动态工作负载的自适应隔离级别选择机制:将工作负载特征建模为依赖图,利用 GNN 学习低隔离级别的性能收益与维持 SER 的开销之间的权衡关系,并实时预测未来工作负载的最优隔离级别(RC, SI 或 SER)。
	\end{itemize}
	\includegraphics[width=0.8\linewidth]{figs/TxnSails-selection-model}
\end{frame}
%%%%%%%%%%%%%%%%%%%%

%%%%%%%%%%%%%%%%%%%%
\begin{frame}{VLDB2025 TxnSails-Achieving Serializable Transaction Scheduling with Self-Adaptive Isolation Level Selection}
	\begin{columns}
		\begin{column}{0.5\textwidth}
			\textbf{Contributions: }
			\begin{itemize}
				\item 跨隔离级别验证机制:当系统根据预测需要切换隔离级别时,检测可能出现的新的危险结构避免破坏 SER;并通过确保提交顺序与依赖顺序一致(必要时中止事务)来保证切换过程中的可串行化,且避免了长时间的阻塞或高中止率。
			\end{itemize}
		\end{column}
		\begin{column}{0.5\textwidth}
			\includegraphics[width=0.98\linewidth]{figs/TxnSails-cross-isolation}
		\end{column}
	\end{columns}
\end{frame}
%%%%%%%%%%%%%%%%%%%%

%%%%%%%%%%%%%%%%%%%%
\begin{frame}{VLDB2025 TxnSails-Achieving Serializable Transaction Scheduling with Self-Adaptive Isolation Level Selection}
\end{frame}
%%%%%%%%%%%%%%%%%%%%