% robustness.tex

%%%%%%%%%%%%%%%%%%%%
\begin{frame}{}
	数据库鲁棒性常被分为动态鲁棒性和静态鲁棒性两个层次来讨论。

	\begin{itemize}
		\item 动态鲁棒性研究数据库的某一个弱隔离级别 $I_1$ 下的执行历史 $H$ 是否满足较强的隔离级别 $I_2$(一般是 SER)。如果满足,我们说 $H$ 在 $I_1$ 下是鲁棒的。
		\item 静态鲁棒性研究客户程序 $P$ 在弱隔离级别 $I_1$ 下的所有执行历史是否都满足较强的隔离级别 $I_2$(一般是 SER)。如果满足,我们说 $P$ 在 $I_1$ 下是鲁棒的。例如,如果 $P$ 在 SI 下鲁棒,则我们可以在 SI 下运行 $P$,这可以在保证执行历史满足 SER 的情况下提高系统性能和吞吐率。
	\end{itemize}
\end{frame}
%%%%%%%%%%%%%%%%%%%%

%%%%%%%%%%%%%%%%%%%%
\begin{frame}{}
	\begin{table}[]
\centering
\caption{$I_{1} \setminus I_{2}:$ Robustness against $I_{1}$ relative to $I_{2}$.}
\renewcommand{\arraystretch}{1.8}
\resizebox{\textwidth}{!}{%
\begin{tabular}{|c||c|c|c|}
	\hline
	$I_1 \setminus I_2$ & \pc & \si & \ser
	\\ \hline\hline
	\ru & & & [Ketsman PODS'2020] (非分布式)充要条件
	\\ \hline
	\rc & & & \incell{[Ketsman PODS'2020] (非分布式)充要条件}{[Vandevoort VLDB'2021] (非分布式)事务模板充要条件}
	\\ \hline
	\ra & & &
	\\ \hline
	\cc & [Bouajjani ESOP'2021] 充要条件
			& [Bouajjani ESOP'2021] 充要条件
			& \incell{[Bernardi CONCUR'2016] 充分条件}{[Beillahi CONCUR'2019] 充要条件}
	\\ \hline
	\pc & / & [Bouajjani ESOP'2021] 充要条件
		  & [Bernardi CONCUR'2016] 充分条件
	\\ \hline
	\psiso & / & [Cerone PODC'2016] 充分条件
				 & [Bernardi CONCUR'2016] 充分条件
	\\ \hline
	\si  & / & / & \incell{[Fekete TODS'2005] 充分条件}{\incell{[Cerone PODC'2016\&JACM'2018] 充分条件}{[Bernardi CONCUR'16] 充分条件}}
	\\ \hline
\end{tabular}
}
\end{table}
\end{frame}
%%%%%%%%%%%%%%%%%%%%

% rp-mil.tex

%%%%%%%%%%%%%%%%%%%%
\begin{frame}{}
	\begin{center}
		\blue{RePMILA:arXiv2025}
		\fig{width = 0.80\textwidth}{figs/rp-mil-arXiv2025}
		\begin{columns}
			\column{0.50\textwidth}
				\pause
				\fig{width = 0.95\textwidth}{figs/repmila-program}
			\column{0.50\textwidth}
				\pause
				\fig{width = 0.90\textwidth}{figs/repmila-bank}
		\end{columns}
		\pause
		\vspace{0.30cm}
		\red{Not support for predicate reads, \textsc{Insert}, \textsc{Delete}}
	\end{center}
\end{frame}
%%%%%%%%%%%%%%%%%%%%
% allocation.tex

%%%%%%%%%%%%%%%%%%%%
\begin{frame}{}
	\begin{center}
		\blue{Allocation:TODS2025}
		\fig{width = 0.80\textwidth}{figs/allocation-tods2025}
		\vspace{-0.30cm}
		\begin{columns}
			\column{0.50\textwidth}
				\fig{width = 0.70\textwidth}{figs/allocation-pg}
			\column{0.50\textwidth}
				\fig{width = 0.60\textwidth}{figs/allocation-oracle}
		\end{columns}
	\end{center}
	\pause
	\vspace{-0.50cm}
	\begin{center}
		\red{Assumption: the set of transactions are \emph{known}}
	\end{center}
\end{frame}
%%%%%%%%%%%%%%%%%%%%

%%%%%%%%%%%%%%%%%%%%
\begin{frame}{}
	\fig{width = 0.80\textwidth}{figs/allocation-iff}
	\begin{columns}
		\column{0.50\textwidth}
			\pause
			\fig{width = 0.85\textwidth}{figs/allocation-alg}
		\column{0.50\textwidth}
			\pause
			\fig{width = 0.90\textwidth}{figs/allocation-opt}
	\end{columns}
\end{frame}
%%%%%%%%%%%%%%%%%%%%
% txnsails.tex

%%%%%%%%%%%%%%%%%%%%
\begin{frame}{}
	\begin{center}
		\blue{TxnSails:VLDB2025}
		\fig{width = 0.80\textwidth}{figs/txnsails-vldb2025}

		\pause
		\fig{width = 0.30\textwidth}{figs/static-dynamic}
	\end{center}
\end{frame}
%%%%%%%%%%%%%%%%%%%%

%%%%%%%%%%%%%%%%%%%%
\begin{frame}{}
	\begin{center}
		\fig{width = 0.60\textwidth}{figs/txnsails-bank}
		\fig{width = 0.70\textwidth}{figs/txnsails-bank-balance}
		\pause
		\red{High abort rate \qquad Non-adaptive}
	\end{center}
\end{frame}
%%%%%%%%%%%%%%%%%%%%

%%%%%%%%%%%%%%%%%%%%
\begin{frame}{}
	\fig{width = 0.60\textwidth}{figs/txnsails-overview}
	\pause
	\red{TS} tries to ensure the \blue{\it commit order}
	is consistent with the \blue{\it dynamic dependency order}.
\end{frame}
%%%%%%%%%%%%%%%%%%%%

%%%%%%%%%%%%%%%%%%%%
\begin{frame}{}
	\fig{width = 0.70\textwidth}{figs/txnsails-civ}
\end{frame}
%%%%%%%%%%%%%%%%%%%%

%%%%%%%%%%%%%%%%%%%%
\begin{frame}{}
	\fig{width = 0.80\textwidth}{figs/txnsails-bank-perf}
\end{frame}
%%%%%%%%%%%%%%%%%%%%