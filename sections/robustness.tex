% robustness.tex

%%%%%%%%%%%%%%%%%%%%
\begin{frame}{What is Robustness?}
	数据库鲁棒性常被分为动态鲁棒性和静态鲁棒性两个层次来讨论。

	\begin{itemize}
		\item 动态鲁棒性研究数据库的某一个弱隔离级别 $I_1$ 下的执行历史 $H$ 是否满足较强的隔离级别 $I_2$(一般是 SER)。如果满足,我们说 $H$ 在 $I_1$ 下是鲁棒的。动态鲁棒性的研究与执行历史验证较为相似。
		\item 静态鲁棒性研究客户程序 $P$ 在弱隔离级别 $I_1$ 下的所有执行历史是否都满足较强的隔离级别 $I_2$(一般是 SER)。如果满足,我们说 $P$ 在 $I_1$ 下是鲁棒的。例如,如果 $P$ 在 SI 下鲁棒,则我们可以在 SI 下运行 $P$,这可以在保证执行历史满足 SER 的情况下提高系统性能和吞吐率。
	\end{itemize}
\end{frame}
%%%%%%%%%%%%%%%%%%%%

%%%%%%%%%%%%%%%%%%%%
\begin{frame}{State of the Art}
	\begin{table}[]
\centering
\renewcommand{\arraystretch}{1.8}
\resizebox{\textwidth}{!}{%
\begin{tabular}{|c|c|c|c|c|c|}
	\hline
	$I_1 \setminus I_2$ & \pc & \si & \ser & \multicolumn{1}{c|}{15-20} & \multicolumn{1}{c|}{总数}
	\\ \hline
	\ru  & & & {[}Ketsman PODS'20{]} (非分布式)充要条件 & \multicolumn{1}{c|}{3} & \multicolumn{1}{c|}{15}
	\\ \hline
	\rc  & & & {[}Ketsman PODS'20{]} (非分布式)充要条件 {[}Vandevoort VLDB'21{]} (非分布式)事务模板的充要条件 & \multicolumn{1}{c|}{0} & \multicolumn{1}{c|}{10}
	\\ \hline
	\ra  & & & & \multicolumn{1}{c|}{3} & \multicolumn{1}{c|}{25}
	\\ \hline
	\tcc  & {[}Bouajjani ESOP'21{]} 充要条件,转化为 CC 对 SER 鲁棒性 & {[}Bouajjani ESOP'21{]} 充要条件,CC \textbackslash{}to PC \textbackslash{}land PC \textbackslash{}to SI & {[}Bernardi CONCUR'16{]} 充分条件 {[}Beillahi CONCUR'19{]} CC、CM、CCv 下充要条件                       &                            &
	\\ \hline
	\pc  & / & {[}Bouajjani ESOP'21{]} 充要条件,依赖程序搜索 & {[}Bernardi CONCUR'16{]} 充分条件 & &
	\\ \hline
	\psiso & / & {[}Cerone PODC'16{]} 充分条件 & {[}Bernardi CONCUR'16{]} 充分条件 & &
	\\ \hline
	\si  & / & /                                                                                                   & {[}Fekete TODS'05{]} 充分条件 {[}Cerone PODC'16 \& JACM'18{]} 充要条件 {[}Bernardi CONCUR'16{]} 充分条件 &                            &                         \\ \hline
\end{tabular}
}
\end{table}
\end{frame}
%%%%%%%%%%%%%%%%%%%%

%%%%%%%%%%%%%%%%%%%%
\begin{frame}{TODS2025 Allocating Isolation Levels to Transactions in a Multiversion Setting}
	background, problems, contributions, techniques (basic ideas), limitations
\end{frame}
%%%%%%%%%%%%%%%%%%%%

%%%%%%%%%%%%%%%%%%%%
\begin{frame}{arXiv2025 2501.18377v3 Using Read Promotion and Mixed Isolation Levels for Performant Yet Serializable Execution of Transaction Programs}
	background, problems, contributions, techniques (basic ideas), limitations
\end{frame}
%%%%%%%%%%%%%%%%%%%%

%%%%%%%%%%%%%%%%%%%%
\begin{frame}{VLDB2025 TxnSails-Achieving Serializable Transaction Scheduling with Self-Adaptive Isolation Level Selection}
	background, problems, contributions, techniques (basic ideas), limitations
\end{frame}
%%%%%%%%%%%%%%%%%%%%