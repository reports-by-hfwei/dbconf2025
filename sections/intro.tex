% intro.tex

%%%%%%%%%%%%%%%%%%%%
\begin{frame}{}
  \begin{center}
    \red{事务}是数据库系统中的核心概念,支撑在线交易顺利进行
  \end{center}

  \begin{columns}
    \column{0.50\textwidth}
      \fig{width = 0.90\textwidth}{figs/db-si}
    \column{0.50\textwidth}
      \fig{width = 0.85\textwidth}{figs/acid}
  \end{columns}

  \vspace{0.30cm}
  \begin{center}
    \red{隔离性} (Isolation) 要求并发事务互不干扰, 避免产生数据异常
  \end{center}
\end{frame}
%%%%%%%%%%%%%%%%%%%%

%%%%%%%%%%%%%%%%%%%%
\begin{frame}{}
  \begin{center}
    强弱不同的隔离级别, 也称\red{事务一致性模型}

    \vspace{0.30cm}
    \fig{width = 0.70\textwidth}{figs/tcm}

    \begin{columns}
      \column{0.15\textwidth}
      \column{0.70\textwidth}
        \begin{description}
          \item[\ru:] Read Uncommitted
          \item[\rc:] Read Committed
          % \item[\ra:] Read Atomicity
          \item[\tcc:] Transactional Causal Consistency
          \item[\si:] Snapshot Isolation
          \item[\ser:] Serializability
        \end{description}
      \column{0.15\textwidth}
    \end{columns}
  \end{center}
\end{frame}
%%%%%%%%%%%%%%%%%%%%

%%%%%%%%%%%%%%%%%%%%
% \begin{frame}{Transaction and Isolation Level}
%   \begin{center}
%     A transaction is a \blue{\it group} of operations
%     that are executed \red{atomically}.

%     \vspace{0.30cm}
% 		% \resizebox{0.55\textwidth}{!}{\input{tikz/isolation-intro-write-skew-tikz}}

%     \vspace{0.20cm}
%     The isolation levels specify how concurrent transactions \\[2pt]
%     are isolated from each other.
%   \end{center}
% \end{frame}
%%%%%%%%%%%%%%%%%%%%

%%%%%%%%%%%%%%%%%%%%
% \begin{frame}{}
%   \begin{center}
%     \red{可串行化 (SER):}
%     所有事务的执行结果与某个串行执行的结果一致

%     \vspace{0.50cm}
% 		\resizebox{0.50\textwidth}{!}{\input{tikz/isolation-ser-write-skew-tikz}}
%   \end{center}
% \end{frame}
%%%%%%%%%%%%%%%%%%%%

%%%%%%%%%%%%%%%%%%%%
% \begin{frame}{}
%   \begin{center}
%     \red{快照隔离 (SI):} 弱于 SER, 允许\blue{\it 写偏斜}数据异常

%     \vspace{0.50cm}
% 		\resizebox{0.50\textwidth}{!}{\input{tikz/isolation-si-write-skew-tikz}}

%     \vspace{0.20cm}
%     \violet{快照读性质:} 每个事务从它开始时的数据库{\it 快照}中读取数据
%   \end{center}
% \end{frame}
%%%%%%%%%%%%%%%%%%%%

%%%%%%%%%%%%%%%%%%%%
% \begin{frame}{}
%   \begin{center}
%     \red{快照隔离 (SI):} 弱于 SER, 但不允许\blue{\it 更新丢失}数据异常

%     \vspace{0.30cm}
%     \resizebox{0.48\textwidth}{!}{\input{tikz/isolation-si-lost-update-tikz}}
%   \end{center}

%   \vspace{-0.50cm}
%   \violet{快照写性质:}
%     并发事务之间不能存在写冲突; 最多只有一个可以提交
% \end{frame}
%%%%%%%%%%%%%%%%%%%%

%%%%%%%%%%%%%%%%%%%%
\begin{frame}{}
  \begin{center}
    事务一致性模型是数据库系统与客户程序之间的一种\red{\bf 契约}

    \vspace{0.30cm}
    \fig{width = 0.60\textwidth}{figs/layers}
  \end{center}
\end{frame}
%%%%%%%%%%%%%%%%%%%%

%%%%%%%%%%%%%%%%%%%%
\begin{frame}{}
  \begin{center}
    以\red{事务一致性模型}为核心的正确性问题至关重要

    \vspace{0.30cm}
    \fig{width = 0.60\textwidth}{figs/correctness}
  \end{center}
\end{frame}
%%%%%%%%%%%%%%%%%%%%

%%%%%%%%%%%%%%%%%%%%
\begin{frame}{}
  \fig{width = 0.65\textwidth}{figs/verification-problems}
\end{frame}
%%%%%%%%%%%%%%%%%%%%